\documentclass{article}

\usepackage{amsmath, mathtools, amsthm}

\usepackage{graphicx}
\graphicspath{ {./images/} }

\title{Networking}
\author{github.com/asdrubalini}
\date{\today}

\begin{document}
    \maketitle

    \section{TCP e UDP}
    TCP: Orientato alla connessione
    UDP: Non richiede una connessione

    Protocollo TFTP su rete locale non poggia su TCP ma su UDP perchè la rete locale ha una bassa latenza e una bassa probabilità di errori.

    Anche il protocollo DNS su rete locale usa UDP, mentre su internet usa TCP.
    Due server DNS tra loro non dialogano un UDP ma in TCP, sempre sulla porta 53.

    Il DNS è un protocollo critico perchè i DNS di tutto il mondo devono parlarsi e scambiarsi informazioni. I regimi dittatoriali configurano i DNS presenti sul territorio in modo che non diano tutte le informazioni.

    Ognuno si deve fidare delle informazioni che gli vengono rilasciate dall'altro.
    Ci sono delle autorità superiori che controllano chi possiede i DNS. Se si verificano delle anomalie, le autorità riescono a risalire a chi ha creato le anomalie.

    Socket = IP + Porta
    L'header del TCP normalmente ha 20 bytes + da 0 a 32 bytes opzionali.

    Sequence number = TCP tiene sotto controllo il flusso e da un ordine ai pacchetti.
    Acknowledgement number = chi riceve i dati comunica a chi li ha trasmessi il corretto recapito dei dati.

    Bit di controllo = quali funzioni sono attive in quel pacchetto. Ad esempio, se il pacchetto è parte di un segmento fragmentato oppure no. Oppure se il pacchetto contiene oppure no un ACK number.

    Window = delimitare il numero massimo di pacchetti che possono essere spediti senza ricevere un ACK.

    Checksum = controllo degli errori nella trasmissione (sia header che dati)

\end{document}
