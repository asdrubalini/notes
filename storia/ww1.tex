\documentclass{article}

\usepackage{amsmath, mathtools, amsthm}
\usepackage{graphicx}
\graphicspath{ {./images/} }

\title{Prima guerra mondiale}
\author{github.com/asdrubalini}
\date{\today}

\begin{document}
\maketitle

\section{Le cause dello scoppio della guerra}
Le cause che hanno contribuito a scatenare la prima guerra mondiale si possono suddividere in quattro categorie:

\subsection{Cause politiche}
\begin{enumerate}
    \item Germania voleva diventare prima potenza mondiale al posto dell'inghilterra;
    \item La Francia voleva prendersi la rivincita contro la Germania;
    \item Crisi dell'impero ottomano: gli altri stati volevano espandersi. La prima a farlo fu l'Austria;
    \item La presenza di due blocchi militari contrapposti: triplice alleanza (Germania, Austria, Italia) e triplice intesa (Gran Bretagna, Francia e Russia).
\end{enumerate}

\subsection{Cause economiche}
\begin{enumerate}
    \item Rivalità tra le colonie da parte delle potenze industriali.
\end{enumerate}

\subsection{Cause militari}
\begin{enumerate}
    \item Corsa agli armamenti spinta dai gruppi industriali.
\end{enumerate}

\subsection{Cause culturali}
\begin{enumerate}
    \item Nazionalismo: volontà di affermare la propria nazione sulle altre;
    \item Applicazione del Darwinismo: la guerra decide le nazioni che dominano;
    \item I giovani borghesi volevano affermarsi grazie alla guerra;
    \item Futurismo, la guerra era la sola igiene del mondo.
\end{enumerate}

\subsection{Lo scoppio della guerra}

Il 28 Giugno 1914 l'arciduca Francesco Ferdinando, erede al trono Austriaco, viene assassinato da un ragazzo Serbo.
L'Austria chiede alla Serbia di consegnargli l'assassino, tuttavia la Serbia è costretta a rifiutare poichè aveva
stretto un accordo con la Russia.\\
Un mese dopo, il 28 Luglio 1914, inizia la guerra tra Austria e Germania contro Serbia, Russia, Francia ed Inghilterra.
Inizialmente si pensa ad una guerra lampo per via delle nuove tecnologie, tuttavia così non fu.\\
\newline
L'Italia si divide in due fazioni: neutralisti ed interventisti.



28 Giugno 1914: arciduca Francesco Ferdinando, erede al trono Austriaco, viene ucciso.
L'Austria chiede un ultimatum alla Serbia chiedendo di consegnare l'assassino. La Serbia respinge poichè aveva stretto
un accordo con la Russia.

28 Luglio 1914: Austria e Germania vs Serbia, Russia, Francia ed inghilterra.
Si pensava ad una guerra lampo per via degli aerei e delle trincee, una guerra di logoramento, ma non fu così.

L'Italia si divide in neutralisti ed interventisti.
Opinione pubblica, liberali (Giolitti), cattolici e la maggior parte dei socialisti non volevano entrare in guerra.
Nazionalisti, irredentisti, gli industriali volevano entrare ed una parte dei socialisti nazionalisti.

Giolitti non voleva che l'Italia entrasse in guerra perchè pensavano che non fossero pronti.

Gli irredentisti volevano coloro che volevano l'unificazione dell'Italia e liberare le terre sotto il dominio Austriaco.
Terre irredente = liberazione dei territori italiani sotto il dominio austriaco.
Affermavano che la guerra sarebbe finita in tre o quattro mesi.
Giolitti invece era convinto che la guerra sarebbe stata lunghissima.
Vincere a fianco dell'Austria per l'Italia non sarebbe servito a nulla.

\subsection{Sabato 12 Febbraio 2022}
L'Italia non era assolutamente preparata ad affrontare un conflitto. L'armamento era pieno di carenze e c'era una forte diversità nelle zone d'Italia. Si trovano a combattere insieme contadini del sud e operai del nord.
\\
Luigi Cadorna era un comandante molto rigido. Disciplina Militare. Si diffonde le fucilazioni in caso di diserzione. Se un soldato prova a scappare viene fucilato. Per quanto riguarda i reati collettivi, si diffonde la pratica della decimazione. 10 uomini venivano estratti a sorte e venivano uccisi.
\\
Fronti della guerra: 4 <slide teams>
\\
Il generale Cadorna tenta un attacco centrale contro l'Austria che porta a migliaia di vittime e nessun territorio.
Gli Austriaci interpretano l'entrata in guerra dell'Italia come un tradimento, nonostante in teoria non lo fosse.

Principali battaglie:
\begin{enumerate}
    \item Battaglia delle Somme
    \item Battaglia dello Jutland
\end{enumerate}

STORIA:
Sono seduti in un campo di papaveri nel fronte occidentale e stanno giocando a carte quando arriva una lettera dal professore che li aveva invogliati ad andare in guerra. Un loro amico si è ferito gravemente ad una gamba ed è in ospedale.

\end{document}
