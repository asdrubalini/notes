\documentclass{article}

\usepackage{amsmath, mathtools, amsthm}
\usepackage{graphicx}
\graphicspath{ {./images/} }

\title{Verga ed i Malavoglia}
\author{github.com/asdrubalini}
\date{\today}

\begin{document}
    \maketitle

    TODOs, [2/1/22 11:28 AM]
Verga sostiene che il progresso sia una fonte di sofferenza. I vinti sono quelli che non riescono a stare a passo con il progresso.
Ideale dell'ostrica: se l'ostrica si stacca dallo scoglio muore. Se noi ci stacchiamo dalla famiglia e dalle tradizioni moriamo. Per non perderci dobbiamo restare ancorati al nostro sistema di valori.

TODOs, [2/1/22 11:28 AM]
Verga per descrivere le classi povere della Sicilia ha bisogno di una serie di documenti. Nasce in Sicilia da una famiglia di proprietari terrieri, non ha mai provato sulla sua pelle quello di cui parla.

TODOs, [2/1/22 11:36 AM]
I malavoglia, dei pescatori siciliani. Protagonisti sono i vinti. Il nome dell'intero ciclo sarebbe stato ciclo dei vinti, avrebbe dovuto rappresentare gli sconfitti nella lotta dell'affermazione sociale.

TODOs, [2/1/22 11:39 AM]
Secondo Verga, al top della scala sociale ci sono gli artisti. I pescatori invece sono al gradino più basso

TODOs, [2/1/22 11:50 AM]
La vicenza si svolge subito dopo l'unità d'Italia 1865-1875 ad Acitrezza in provincia di Catania. 

Bastianaccio aveva sposato Maruzza la Longa

TODOs, [2/1/22 5:50 PM]
Il giovane è costretto a partire per la leva militare. Padron ntoni decide di comprare a debito dei lupini per poi rivenderli. La barba però naufraga e muoiono in mare. Vende la barca e vende la casa per ripagare il debito dei lupini.


\end{document}
