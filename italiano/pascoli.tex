\documentclass{article}

\usepackage{amsmath, mathtools, amsthm}
\usepackage{graphicx}
\graphicspath{ {./images/} }

\title{Pascoli}
\author{github.com/asdrubalini}
\date{\today}

\begin{document}
\maketitle

\section{Pascoli}
Nasce a San Mauro di Romagna in una famiglia borghese perchè suo padre amministrava una grande tenuta di proprietà dei principi
di torronia. Suo padre muore nel 1867 per colpa di una fucilata. Successivamente, anche sua sorella, sua madre ed i suoi due
fratelli morirono. Per questo motivo si fissò con l'idea di ricostruire il nido familiare ed ebbe anche un rapporto morboso
con la sorella.\\
Si laurea a lettere classiche e insegna sia al liceo che all'università, ottenendo la cattedra di Carducci. Era un nazionalista
con Mussolini.

\subsection{Stile letterario}
Le poesie di Giovanni Pascoli sono caratterizzate da frasi brevi e periodi spezzati e coordinati. I termini che utilizza
sono di uso comune, talvolta anche dialettali. Utilizza anche molte onomatopee, ovvero i suoni come parole. Abbandona la forma
chiusa del sonetto ma rifiuta anche di utilizzare una struttura libera, soffermandosi su una grande varietà di metriche
regolari, rime, pause ed emjambement (rottura della metrica in cui il senso si prolunga nel verso successivo).\\
Il poeta combina esperienze sensoriali differenti. Rifiuta la rappresentazione realistica ma bensità abbraccia quella decadentista
(rappresentazione della realtà attraverso l'arte e non più la scienza, si rappresenta la realtà interiore e non più quella esteriore,
abbandona la ragione).\\

Raccolte poetiche:
\begin{itemize}
    \item Myricae: insieme di Poesie
    \item Poemetti
    \item Canti di Castelvecchio
\end{itemize}

\section{Testi}
\subsection{Myricae}
Nella prefazione di Myricae vengono trattati due temi: la morte (riferito a suo padre) e la natura.

Primo testo di Pascoli: Lavandare, appartenente alla raccolta Myricae, riporta al canto delle lavandaie.\\
Tema dell'incompletezza e dell'abbandono: immagine di un campo mezzo arato e mezzo no.\\
Tema della perdita: immagine di un aratro che pare abbandonato. Figura retorica della litterazione, ovvero della
ripetizione di parole che iniziano con la stessa lettera.\\
Troviamo l'uso del senso della vista e l'uso del senso dell'udito. "vedere e sentire, altro non deve il poeta".\\
Successivamente viene rivelato il significato simbolico del componimento. Il campo arato a metà, l'aratro dimenticato,
la nebbia che impedisce di vedere e l'autunno che è la stagione della morte.

\subsection{X Agosto}
10 Agosto. La poesia è dedicata alla morte del padre che Pascoli collega a quella di una rondine. Viene pubblicata 30 anni dopo.\\
Nella prima quartina il cielo è come se stesse piangendo di stelle cadenti (figura retorica).\\
Nella seconda quartina il tetto è una metafora per dire nido, una metafora umanizzante. Insiste sui suoni per comunicare
un'emozione. Si parla della morte della rodine. Figura retorica della sineddoche, in cui si usa il contenitore per indicare
il contenuto.\\
quest atomo opaco del male = la Terra
Nella terza quartina la rondine rappresenta simbolicamente il crocifisso. Il ciel è indifferente, non si cura della morte.\\
Il componimento ha una struttura circolare perchè la prima e l'ultima quartina coincidono.
Parallelismo: Solitudine dei propri figli. Indifferenza del cielo, metafora tetto, nido. Non specifica mai che si tratta di
suo padre perchè vuole comunicare un dolore universale, la sofferenza degli uomini. Paragone con Cristo. Sono entrambi
morti innocenti come la morte di cristo. Unica differenza è che qui non c'è nessun avvicinamento tra la terra e il cielo ma
anzi viene visto lontano.

\subsection{Figure retoriche}
Figura retorica che Pascoli usa molto: enjambemont e onomatopea.
Dea Iside è sia la dea della morte che della resurrezione. Gli egiziani utilizzavano questi strumenti in onore della dea Iside.
Cosa importante: climax ascendente. Si alternano immagini di serenità ad immagini.
Nella seconda strofa sente risuonare il grido della morte del padre. Un concetto fondamentale di pascoli è che quello che
descrive ha un significato simbolico e non solo realistico e significativo.

\subsection{Temporale}
Riferimento al cielo, versi settenari brevi e spezzati.\\
C'è la paratassi per asindeto, ovvero le frasi si sostengono con la virgola al posto della congiunzione.\\
Le frasi nominali: \\
C'è una presenza di onomatopee, ad esempio bubbolio che richiama il suono del tuono.\\
Una contrapposizione tra una parte negativa e malinconica (fuori clima minaccioso) e una parte serena (casolare).\\
Le ali di gabbiano danno l'idea di un senso di protezione e di un senso di fuga.\\
Pascoli parla per immagini

\subsection{}
PASCOLI: temporaneo.
Poesia: temporale
Arrivo contemporaneo della pianura
Frammentismo: versi brevi e spezzati
Paratassi per asindeto: le frasi si susseguono attraverso la virgola
Versi corti. Ala di gabbiano = paragone.
La parte di serenità è il casolare. La casa rappresenta il nido. All'esterno c'è un clima minaccioso e lui si rifugia all'interno.
Pascoli parla anche per immagini. Simbolismo dei colori. Bianco è il colore della speranza e il nero ed il rosso sono i colori
dell'angoscia. Sensorialità: quali dei cinque sensi pascoli utilizza per descrivere la poesia. In questo caso vista ed udito.
Paratassi per polisindeto

\end{document}
