\documentclass{article}

\usepackage{amsmath, mathtools, amsthm}
\usepackage{graphicx}
\graphicspath{ {./images/} }

\title{Positivismo}
\author{github.com/asdrubalini}
\date{\today}

\begin{document}
    \maketitle

    \section{Positivismo}
    Periodo nel quale si afferma una sorta di nuova religione.\\
    L'uomo diventa artefice indiscusso del proprio progresso.
    Tutto ciò che riguarda la metafisica non deve essere oggetto dello studio dell'indagine filosofica. La filosofia mette in discussione la ragione.

    Secondo Comte:
    
    "Basta occuparsi di metafisica, basta pensare ai massimi sistemi. Indagine filosofica deve essere un'indagine scientifica. No questioni indecifrabili. Più argomenti accessibili che riguardano il reale.".\\
    Dovrebbe cercare il miglioramento delle condizioni dell'uomo.\\
    Secondo lui la filosofia sono inutili dubbi.\\
    Nel pensiero dell'uomo esiste una sorta di evoluzionismo Darwiniano.

    Speculazione, ma non quella economica.
    La filosofia serve per dare certezze, e le certezze le da la scienza e la tecnica.

    Comte - Darwin - Nietzsche

    L'uomo è profondamente determinato a seconda dell'ambiente in cui vive.\\
    L'uomo ha il gusto per il dolce perchè lo portiamo nel DNA che affonda le sue origini dalla preistoria.\\
    Due entità nella storia: individuo e società.

    Nietzsche: "Non esiste una razionalità nella storia."\\
    Superuomo per Nietzsche è colui che dopo aver lottato ed essersi imposto, regnerà sulla terra. Dominerà la terra ed i soggetti più deboli dopo la morte di Dio. La fine dei vecchi valori. Imparerà a liberarsi dalla morale. Non esiste più il bene e il male. Autodeterminazione dell'uomo ed una conseguente affermazione del diritto di questo superuomo di fare quello che vuole.

\end{document}
