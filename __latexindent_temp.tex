\documentclass{article}

\usepackage{amsmath, mathtools, amsthm}
\usepackage{graphicx}
\graphicspath{ {./images/} }

\title{Leopardi}
\author{github.com/asdrubalini}
\date{\today}

\begin{document}
    \maketitle

    \section{Il Sabato del villaggio (1828)}
    La prima strofa è molto descrittiva.
    L'Idillio nella poesia greca è un componimento potetico di carattere pastorale.
    Leopardi prende spunto dalla poesia greca ma la amplia aggiungendo una sua riflessione sulla condizione esistenziale dell'uomo.
    \newline
    L'uomo confonde la speranza con i suoi desideri. Ognuno attraverso il suo intelletto farà ritorno al faticoso lavoro abituale.
    Man mano che passano le ore della domenica cresce l'ansia perchè ognuno torna al suo lavoro abituale. Poi si rivolge in maniera diretta
    a te, un giovane del 2021 che sta leggendo il testo e che non conosce ancora il male della vita. Tutti siamo in attesa di qualcosa, però
    quando quel qualcosa arriva effettivamente, ne restiamo delusi.
    La festa di cui parla Leopardi è l'età adulta, uscire di casa, andare via dall'oppressione dei propri genitori che impongono delle regole,
    avere un'indipendenza economica.
    \newline*
    Sembra un qualcosa di pesante, l'oppressione della scuola, l'ITIS che è un carcere, l'oppressione sistematica da parte dei genitori, senza
    rendersi conto che si sta vivendo in realtà il momento più bello, il sabato della vita. $\langle \langle$Ma sappi una questione, la tua festa (età adulta)
    anche se dovesse tardare a venire, non ti sia pesante.$\rangle \rangle$. Arriverà la festa e vi accorgerete che sarà tutt'altro.
    Da una descrizione ad una riflessione sulla vita di carattere generale.
    \newline
    Meditare fa rima con medicare, l'etimologia è la stessa. Le conclusioni di Leopardi appaiono corrette, pur nella loro malinconica concezione
    della vita dettata dal suo pragmaticismo materialista.
    \newline
    Esiste la felicità o vi è attesa di qualcosa che possa illuderci dell'attesa della felicità?
    Dolce illusione dei caratteri giovanili. La festa vera non è nella domenica ma nel sabato che la precede. Viene fuori il materialismo.
    Una grande similitudine con giovinezza (età della presunzione) ed età adulta. Uno pensa di comprendere tutto della vita.
    $\langle \langle$Fate si che la vostra giovinezza trascorra lentamente.$\rangle \rangle$
    \newline
    Condizione di pessimismo individuale.
    Idillio: componimento poetico di carattere pastorale / contadino + riflessione sulla condizione esistenziale dell'uomo.
    In Greco significa piccolo quadro / piccola immagine. Anche per Leopardi diventano un quadro sull'immagine della natura, quasi la
    contemplasse.
    Per l'infinito, molti critici hanno parlato addirittura di Idillio sacro. In fondo vi è comunque un'intuizione religiosa.
    Leopardi va oltre la natura. Dalla contemplazione si passa alla medicazione.
    \newline
    Che cos'è la vita per Leopardi? Lo troviamo nello Zibaldone, il suo diario personale. Lo Zibaldone non è un'opera vera e propria,
    ed è uscita dopo la sua morte. 15 anni di riflessioni, meditazioni, riflessioni su se stesso e gli altri, riflessioni sulla vita.
    Vien fuori la sua vita. Cantiere dove butta giù le idee. Non ha una logica, non si può leggere come se fosse.
    \newline
    Leopardi: $\langle \langle$Che cos'è la vita?$\rangle \rangle$. E poi parla di tutt'altro. Lo spiega attraverso una metafora. $\langle \langle$Il viaggio di uno zoppo infermo che con
    un gravissimo carico in sul dorso per montagne altissime e luoghi sommamente aspri, faticosi e difficili, alla neve, al gelo,
    alla pioggia, al vento, all'ardore del Sole, cammina senza mai riposarsi, dì e notte, uno spazio di molte giornate per arrivar a un
    cotal precipizio o fosso e quivi inevitabilmente cadere$\rangle \rangle$ (1826).
    Quello che i critici chiamano pessimismo cosmico.
    \newline
    Viaggio di una persona che soffre ed è infermo. La malattia dell'uomo è il materialismo, che però non viene detto esplicitamente.
    Il gravissimo carico che ogni uomo ha sul dorso è la ragione, che ci nobilita e ci può condannare se non correttamente intesa, come un dispetto.
    La vita è un viaggio in salita. Uno si accorge che sta vivendo non quando va in discesa ma quando inizia la salita ed inizia a scoprire tutte
    i problemi che gli si prospettano davanti. Non esiste un uomo che può dirsi esente dal far fatica. Neanche i più potenti al mondo.
    Tutti facciamo esperienza della fatica e di momenti difficili. Sia al gelo, sia quando tutto è ghiacciato.
    Camminiamo senza mai riposarci per arrivare di fronte ad un precipizio e cadere inevitabilmente dentro.
    Parte da una concezione razionale.
    \newline
    Se la vita ha senso, Leopardi ha torto. Ma se la vita non ha senso, ha ragione. La vita è solo materia oppure c'è altro?
    Un dono che non può essere eterno. Perchè abbiamo dentro di noi la coscenza dell'infinito e della realtà? (min 59) Pensare a qualcosa
    di eterno.

\end{document}
