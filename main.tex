\documentclass{article}

\usepackage{amsmath, mathtools, amsthm}

\title{Onde elettromagnetiche}
\author{github.com/asdrubalini}
\date{\today}

\begin{document}
    \maketitle
    
    \section{Definizione del campo elettrico E}

    \begin{equation}
        \bar{E}^+(x, t) = E^+_M \cdot e^{j\omega (t-\frac{x}{u})}
    \end{equation}

    Con $k$ si indica la costante di fase che è definita con seguente rapporto

    \begin{equation}
        k = \frac{\omega}{u}
    \end{equation}

    quindi

    \begin{equation}
        \bar{E}^+(x, t) = E^+_M \cdot e^{j (\omega t - kx)}
    \end{equation}

    \section{Definizione del campo magnetico H}    

    \section{Velocità di onde elettromagnetiche}

    La velocità di un'onda elettromagnetica è costante e si può calcolare con

    \begin{equation}
        u = \frac{1}{\sqrt{\varepsilon \mu}} \hspace{0.2cm} [m/s]
    \end{equation}

    dove $\mu$ è la permeabilità magnetica e $\varepsilon$ è la permittività elettrica.
    Nel vuoto, la formula diventa

    \begin{equation}
        c = \frac{1}{\sqrt{\varepsilon_0 \mu_0}} = 299 792 458 \hspace{0.2cm} [m/s]
    \end{equation}

    \vspace{1cm}

    La permeabilità magnetica si può esprimere come il prodotto

    \begin{equation}
        \mu = \mu_0 \mu_r
    \end{equation}

    dove $\mu_0$ è la permeabilità magnetica del vuoto e $\mu_r$ è la permeabilità magnetica relativa
    del materiale.

    \vspace{1cm}

    Allo stesso modo, la permittività elettrica si può esprimere come il prodotto

    \begin{equation}
        \varepsilon = \varepsilon_0 \varepsilon_r
    \end{equation}

    dove $\varepsilon_0$ è la permittività elettrica del vuoto e $\varepsilon_r$ è la permittività elettrica relativa
    del materiale.

    \section{Impedenza caratteristica}

    L'impedenza caratteristica del vuoto è costante e si può calcolare con

    \begin{equation}
        Z_0 = \sqrt{\frac{\mu_0}{\varepsilon_0}} = c_0 \mu_0 \approx 377 \hspace{0.2cm} [\Omega]
    \end{equation}

    In generale, l'impedenza caratteristica di un mezzo diverso dal vuoto si calcola con

    \begin{equation}
        Z = \sqrt{\frac{\mu}{\varepsilon}} = c \mu = Z_0 \frac{u}{c} \hspace{0.2cm} [\Omega]
    \end{equation}

    \begin{equation}
        Z = Z_0 \sqrt{\frac{\mu_r}{\varepsilon_r}} \hspace{0.2cm} [\Omega]
    \end{equation}

    \section{Lunghezza d'onda, frequenza e altre caratteristiche di un'onda}

    Con la lettera greca lambda si rappresenta la lunghezza d'onda che si misura in metri.

    \begin{equation}
        \lambda = u T = \frac{u}{f} \hspace{0.2cm} [m]
    \end{equation}

    \begin{equation}
        \lambda = \frac{\omega}{k f}
    \end{equation}

    Da cui ricaviamo che la costante di fase (k) si può calcolare, sapendo la lunghezza d'onda, con:

    \begin{equation}
        k = \frac{2 \pi}{\lambda}
    \end{equation}

    \section{Densità di potenza}
    
    Il campo elettromagnetico, essendo composto da una part di campo elettrico (E) ed una parte di campo magnetico
    (H), ha associata una potenza S che rappresenta l'energia che nell'unità di tempo attraversa una sueprficie.

    \begin{equation}
        S = \frac{1}{2} E_M H_M
    \end{equation}

    Dove $H_M = \frac{E_M}{\zeta}$, quindi:

    \begin{equation}
        S = \frac{1}{2} \frac{{E_M}^2}{\zeta} \hspace{0.2cm} [\frac{W}{m^2}]
    \end{equation}

\end{document}
